\iffalse
  \title{APPLICATIONS OF DERIVATIVES}
  \author{HARSHILL RATHAN}
  \section{mains}
\fi

%   \begin{enumerate}
    \item The maximum distance from origin of a point on the curve \begin{align*}
    x = a \sin\brak{t-b} \sin\brak{\frac{at}{b}}
\end{align*}
\begin{align*}
     y = a \cos\brak{t-b} \cos\brak{\frac{at}{b}}
\end{align*} 
both $a, b > 0$ \hfill{[2002]}
\begin{multicols}{2}
\begin{enumerate}
    \item $a-b$
    \item $a+b$
    \item $\sqrt{a^2+b^2}$
    \item $\sqrt{a^2-b^2}$ 
\end{enumerate}
\end{multicols}
\item If $2a+3b+6c=0$, $\brak{a,b,c \in R}$ then the quadratic equation $ax^2+bx+c$ has \hfill{[2002]}
\begin{multicols}{2}
\begin{enumerate}
    \item at least one root in $[0,1]$
    \item at  least one root in $[2,3]$
    \item at least one root in $[4,5]$
    \item none of these \\
\end{enumerate}
\end{multicols}
\item If the function $f\brak{x}=2x^3-9ax^2+12a^2x+1$, where $a>0$, attains its maximum and minimum at $p$ and $q$ respectively such that $p
    ^2=q$, then $a$ equals  \hfill{[2003]}
\begin{multicols}{2}
\begin{enumerate}
       \item $\frac{1}{2}$
       \item $3$
       \item $1$
       \item $2$
\end{enumerate}
\end{multicols}
\item A point on the parabola $y^2=18x$ at which the ordinate increases at twice the rate of the abscissa is \hfill{[2004]}
\begin{multicols}{2}
\begin{enumerate}
    \item $(\frac{9}{8},\frac{9}{2})$
    \item $\brak{2,-4}$
    \item $(\frac{-9}{8},\frac{9}{2})$
    \item $\brak{2,4}$
\end{enumerate}
\end{multicols}
\item A function $y=f\brak{x}$ has a second order derivative $f"\brak{x}=6\brak{x-1}$.If its graph passes through the point $\brak{2,1}$ and at that point the tangent to the graph $y=3x-5$, then the function is \hfill{[2004]}
\begin{multicols}{2}
\begin{enumerate}
    \item $\brak{x+1}^2$
    \item $\brak{x-1}^3$
    \item $\brak{x+1}^3$
    \item $\brak{x-1}^2$
\end{enumerate}
\end{multicols}
\item The normal to the curve $x=a\brak{1+\cos\theta}$, $y=a \sin\theta$ at $\theta$ always passes through the fixed point\\ \hfill{[2004]}
\begin{multicols}{2}
\begin{enumerate}
    \item $\brak{a,a}$
    \item $\brak{0,a}$
    \item $\brak{0,0}$
    \item $\brak{a,0}$
\end{enumerate}
\end{multicols}
\item If $2a+3b+6c=0$, then atleast one root of the equation $ax^2+bx+c$ lies in the interval \hfill{[2004]}
\begin{multicols}{2}
\begin{enumerate}
    \item $\brak{1,3}$
    \item $\brak{1,2}$
    \item $\brak{2,3}$
    \item $\brak{0,1}$
\end{enumerate}
\end{multicols}  
 \item Area of the greatest rectangle that can be inscribed in the ellipse $\frac{x^2}{a^2}+\frac{y^2}{b^2}=1$is
\begin{multicols}{2}
\begin{enumerate}
    \item $2ab$
    \item $ab$
    \item $\sqrt{ab}$
    \item $\frac{a}{b}$
\end{enumerate}
\end{multicols}
\item The normal to the curve $x=a \cos \theta$+ $\sin\theta$, y=$a\sin\theta - \cos \theta$ at any point $\theta$ is such that
\begin{multicols}{2}
\begin{enumerate}
    \item it passes through the origin
    \item it makes an angle $\frac{\pi}{2}+\theta$ with the x axis
    \item it passes through $(a\frac{\pi}{2},-a)$
    \item it is at a constant distance from the origin
\end{enumerate}
\end{multicols}


% \end{enumerate}
