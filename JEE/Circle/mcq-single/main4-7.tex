\iffalse
\title{Assignment 1}
\author{EE24Btech11024 - G.Abhimanyu Koushik}
\section{mcq-single}
\fi
\item Circle(s) touching the x-axis at a distance \brak{3} from the origin and having an intercept of length $2\sqrt{7}$ on the y-axis is \brak{\text{are}}

\hfill{\brak{JEE Adv. 2013}}
\begin{enumerate}
\begin{multicols}{2}
\item $x^2 + y^2 - 6x + 8y + 9 = 0$
\item $x^2 + y^2 - 6x + 7y + 9 = 0$
\item $x^2 + y^2 - 6x - 8y + 9 = 0$
\item $x^2 + y^2 - 6x - 7y + 9 = 0$
\end{multicols}
\end{enumerate}
\item A circle $S$ passes through the point \brak{0,1} and is orthogonal to the circle $(x-1)^2+y^2=16$ and $x^2+y^2=1$. Then

\hfill {\brak{JEE Adv. 2014}}
\begin{enumerate}
\begin{multicols}{2}
\item Radius of $S$ is $8$
\item Radius of $S$ is $7$
\item Centre of $S$ is \brak{-7,1}
\item Centre of $S$ is \brak{-8,1}
\end{multicols}
\end{enumerate}
\item Let $RS$ be the diameter of the Circle $x^{2} + y^{2} = 1$, where $\vec{S}$ is the point \brak{1,0}. Let $\vec{P}$ be a variable point \brak{\text{other than R and S}} on the circle and tangents to the circle at $\vec{S}$ and $\vec{P}$ meet at the point $\vec{Q}$. The normal to the circle at $\vec{P}$ intersects a line drawn through $\vec{Q}$ parallel to $RS$ at point $\vec{E}$. Then the locus of $\vec{E}$ passes through the point\brak{\text{s}}

\hfill {\brak{JEE Adv. 2016}}
\begin{enumerate}
\begin{multicols}{2}
	\item \brak{\frac{1}{3}, \frac{1}{\sqrt{3}}}
	\item \brak{\frac{1}{4}, \frac{1}{2}}
	\item \brak{\frac{1}{3}, -\frac{1}{\sqrt{3}}}
	\item \brak{\frac{1}{4}, -\frac{1}{2}}
\end{multicols}
\end{enumerate}
\item Let $T$ be a line passing through the points $\vec{P}$\brak{-2,7} and $\vec{Q}$\brak{2,-5}. Let $F_1$ be the set of all pairs of circles \brak{S_1,S_2} such that $T$ is tangent to $S_1$ at $\vec{P}$ and tangent to $S_2$ at $\vec{Q}$, and also such that $S_1$ and $S_2$ touch each other at a point, say $\vec{M}$. Let $E_1$ be the set representing the locus of $\vec{M}$ as the pair \brak{S_1,S_2} varies in $F_1$. Let the set of all straight line segments joining a pair of distinct points of $E_1$ and passing through the point $\vec{R}$\brak{1,1} be $F_2$. Then which of the following statements is (are) TRUE?

\hfill{\brak{JEE Adv. 2018}}
\begin{enumerate}
\begin{multicols}{2}
\item The point \brak{-2,7} lies on $E_1$
\item The point \brak{\frac{4}{5}, \frac{7}{5}} does \textbf{NOT} lie on $E_1$
\item The point \brak{\frac{1}{3},1} lies on $E_1$
\item The point \brak{0, \frac{3}{2}} does not lie on $E_1$
\end{multicols}
\end{enumerate}
