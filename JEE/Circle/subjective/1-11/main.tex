\iffalse
\title{Assignment 1}
\author{EE24Btech11024 - G.Abhimanyu Koushik}
\section{subjective}
\fi
\item Find the equation of the circle whose radius is 5 and which touches the circle $x^2+y^2-2x-4y-20=0$ at the point \brak{5,5}

\hfill {\brak{1978}}
\item Let $\vec{A}$ be the centre of circle $x^2+y^2-2x-4y-20=0$. Suppose that the tangents at the points $\vec{B}$\brak{1,7} and $\vec{D}$\brak{4,-2} on the circle meet at point $\vec{C}$. Find the area of the quadrilateral $ABCD$.

\hfill {\brak{1981 - 4 marks}}
\item Find the equations of the circle passing through \brak{-4,3} and touching the lines $x+y=2$ and $x-y=2$

\hfill {\brak{1981 - 4 marks}}
\item Through a fixed point \brak{h,k} secants are drawn to the circle $x^2+y^2=r^2$. Show that the locus of the mid-points of the secants intercepted is $x^2+y^2=hx+ky$

\hfill {\brak{1983 - 5 marks}}
\item The abscissa of two points $\vec{A}$ and $\vec{B}$ are roots of the equation $x^2+2ax-b^2=0$ and their ordinates are roots of the equation $x^2+2px-q^2=0$. Find the equation and the radius of the circle with $AB$ as diameter.

\hfill {\brak{1984 - 4 marks}}
\item Lines $5x+12y-10=0$ and $5x-12y-40=0$ touch a Circle $C_1$ of diameter 6. If the centre of $C_1$ lies in the first quadrant, find the equation of circle $C_2$ which is concentric with $C_1$ and cuts intecepts of length 8 on these lines

\hfill {\brak{1986 - 5 marks}}
\item Let a given Line $L_1$ intersects the $x$ and $y$ axes at $\vec{P}$ and $\vec{Q}$ respectively. Let another line $L_2$, perpendicular to $L_1$, cut the $x$ and $y$ axes at $\vec{R}$ and $\vec{S}$, respectively. Show that the locus of the point of intersection of $PS$ and $QR$ is a circle passing through origin.

\hfill {\brak{1987 - 3 marks}}
\item The circle $x^2+y^2-4x-y+4=0$ is inscribed in a triangle which has two of its sides along the co-ordinate axes. The locus of circumcentre of the triangle is $x+y-xy+k(x^2+y^2)\textsuperscript{1/2}$. Find $k$.

\hfill {\brak{1987 - 4 marks}}
\item If $\brak{ m_i, \frac{1}{m_i}}, m_i > 0, i = 1, 2, 3, 4$ are four distinct points on a circle, then show that $m_1m_2m_3m_4=1$

\hfill {\brak{1989 - 2 marks}}
\item A circle touches the line $y=x$ at a point $\vec{P}$ such that $OP=4\sqrt{2}$ , where O is the origin. The circle contains the point \brak{-10,2} in its interior and the length of its chord on the line $x+y=0$ is $6\sqrt{2}$. Determine the equation of circle.

\hfill {\brak{1990 - 5 marks}}
\item Two circles, each of radius 5 units, touch each other at \brak{1,2}. If the equation of common tangent is $4x+3y=10$, find the equations of circles.

\hfill {\brak{1991 - 4 marks}}

