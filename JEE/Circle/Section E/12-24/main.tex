\iffalse
  \title{Assignment1 EE24BTECH11025}
  \author{GEEDI HARSHA VARDHAN}
  \section{subjective}
\fi


  % \begin{enumerate}
    
	\item Let a circle be given by $2x\brak{x-a}+y\brak{2y-b}=0$,$\brak{a\neq0,b\neq0}$.Find the condition on $a$ and $b$ if two chords, each bisected by the x-axis,can be drawn to the circle from $\brak{a,\frac{b}{2}}$.                         

\hfill(1992- 6 Marks)\\




\item Consider a family of circles passing through two fixed points A$\brak{3,7}$ and B$\brak{6,5}$. Show that chords in which the circle $x^2+y^2-4x-6y-3=0$ cuts the members of the family are concurrent at a point. Find the coordinate of this point.
	        
\hfill(1993- 5 Marks)\\





\item Find the coordinates of the point at which the circles $x^2+y^2-4x-2y=-4$ and $x^2+y^2-12x-8y=-36$ touch each other. Also find equations common tangents touching the circles in the distinct points.                        

\hfill(1993- 5 Marks)\\


\item Find the intervals of values of a for which the line $y+x=0$ bisects two chords drawn from a point $\brak {\frac{1+\sqrt{2}a}{2},\frac{1-\sqrt{2}a}{2}}$ to the circle $2x^2+2y^2-\brak{1+\sqrt{2}a}x-\brak{1-\sqrt{2}a}y=0$.  

\hfill(1996- 5 Marks)\\





\item A circle passes through three points A,B and C with the line segment AC as its diameter. A line passing through A intersects the chord BC at point D inside the circle. If angles $DAB$ and $CAB$ are $\alpha$ and $\beta$ respectively and the distance between the point A and midpoint of the line segment DC is $d$, prove that the area of the circle is $\frac{\pi d^2 \cos^2{\alpha} }{\cos^2{\alpha}+\cos^2{\beta}+ 2\cos{\alpha} \cos{\beta} \cos{\brak{\beta-\alpha}}}$                

\hfill(1996- 5 Marks)\\





\item Let $C$ be any circle with centre $\brak{0,\sqrt{2}}$. Prove that at the most two rational points can be there on $C$.(A rational point is a point both of whose coordinates are rational numbers)
	           
\hfill(1997- 5 Marks)\\




\item $C_{1}$ and $C_{2}$ are two concentric circles, the radius of $C_{2}$ being twice that of $C_{1}$. From a point $P$ on $C_{2}$, tangents $PA$ and $PB$ are drawn to $C_{1}$. Prove that the centroid of the triangle $PAB$ lies on $C_{1}$.
	           \hfill(1998- 8 Marks)\\




\item Let $T_{1}$, $T_{2}$ be two tangents drawn from $\brak{2,0}$ onto the circle $C$:$x^2+y^2=1$. Determine the circles touching $C$ and having $T_{1}$, $T_{2}$ as their pair of tangents. Further, find the equations of all possible common tangents to these circles, when taken two at a time.
                  \hfill(1999- 10 Marks)\\




\item Let $2x^2+y^2-3xy=0$ be the equation of pair of tangents drawn from the origin $O$ to a circle of radius 3 with the centre in the first quadrant. If $A$ is one of the points of contact, find the length of $OA$.                   \hfill(2001- 5 Marks)\\




\item Let $C_{1}$ and $C_{2}$ be two circles with $C_{2}$ lying inside $C_{1}$. A circle $C$ lying inside $C_{1}$ touches $C_{1}$ internally and $C_{2}$ externally. Identify the locus of centre of $C$.                                \hfill(2001- 5 Marks)\\



\item For the circle $x^2+y^2=r^2$, find the value of $r$ for which the area enclosed by the tangents drawn from the point $P$$\brak{6,8}$ to the circle and the chord of contact is maximum.


\hfill(2003- 2 Marks)\\ 





\item Find the equation of circle touching the line $2x+3y+1=0$ at $\brak{1,-1}$ and cutting orthogonally the circle having line segment joining \brak{0,3} and \brak{-2,-1} as diameter.


\hfill(2004- 4 Marks)\\     




\item Circles with radii 3,4 and 5 touch each other externally. If $P$ is the point of intersection of tangents to these circles at their points of contact, find the distance of $P$ from the points of contact.
	           \hfill(2005- 2 Marks)\\ 
  % \end{enumerate}

