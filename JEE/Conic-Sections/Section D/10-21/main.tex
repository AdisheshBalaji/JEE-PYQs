\iffalse
  \title{ASSIGNMENT-1}
  \author{EE24BTECH11032- JOHN BOBBY}
  \section{mcq-multiple}
\fi


  % \begin{enumerate}
	\item Let $\vec{A}$ and $\vec{B}$ be two distinct points on the parabola $y^2=4x$. If
      	      the axis of the parabola touches a circle of radius r having
		$AB$ as diameter, then the slope of the line joining $\vec{A}$ and $\vec{B}$
	     can be 
		\hfill(2010)
		
		 \begin{enumerate}
			\item$\frac{-1}{r}$
			\item$\frac{1}{r}$
			\item$\frac{2}{r}$
			\item$\frac{2}{r}$
	        \end{enumerate}
	\item Let the eccentricity of the hyperbola $\frac{x^2}{a^2}-\frac{y^2}{b^2}=1$ be reciprocal to that of the elipse $x^2+4y^2=4$. If the hyperbola
	passes through a focus of the elipse, then 
		\hfill(2011)
		
		 \begin{enumerate}
			\item the equation of the hyperbola is $\frac{x^2}{3}-\frac{y^2}{2}=1$
			\item a focus of the hypebola is $(2,0)$
			\item the eccentricity of the hyperbola is $\sqrt{\frac{5}{3}}$
			\item the equation of the hyperbola is $x^2-3y^2=3$
		 \end{enumerate}
	\item Let L be a normal to the parabola $y^2=4x$. If L passes through the point $(9,6)$, then the L is given by 
		\hfill(2010)
		
		 \begin{enumerate}
			\item $y-x+3=0$
			\item $y+3x-33=0$
			\item $y+x-15=0$
			\item $y-2x+12=0$
		 \end{enumerate}
	\item Tangents are drawn to the hyperbola $\frac{x^2}{9}-\frac{y^2}{4}=1$, parallel to the straight line $2x-y=1$. The points of contact of the tangents to the hyperbola
		are  
		\hfill(2012)
		
		 \begin{enumerate}
			\item $\brak{\frac{9}{2\sqrt{2}},\frac{1}{\sqrt{2}}}$ 
			\item $ \brak{\frac{-9}{2\sqrt{2}},\frac{-1}{\sqrt{2}}}$
			\item $\brak{3\sqrt{3},-2\sqrt{2}}$
			\item $ \brak{-3\sqrt{3},2\sqrt{2}}$
		
		 \end{enumerate}

	\item Let $\vec{P}$ and $\vec{Q}$ be distinct points on the parabola $y^2=2x$ such 
		that a circle with $PQ$ as diameter passes through the vertex
		$\vec{O}$ of the parabola. If $\vec{P}$ lies in the first quadrant and the area
		of the triangle  \(\Delta \)$OPQ$ is 3$\sqrt{2}$, then which of the following is
		(are) the coordinates of $\vec{P}$?  
		\hfill(JEE ADV.2015)
		
		 \begin{enumerate}
			\item $ \brak{4,2\sqrt{2}}$
			\item $\brak{9,3\sqrt{2}}$
			\item $ \brak{\frac{1}{4},\frac{1}{\sqrt{2}}}$
			\item  $ \brak{1,\sqrt{2}}$
		 \end{enumerate}
	\item Let $ E_1$  and $ E_2$ be two elipses whose centres are at the orgin.
              The major axes of $E_1$ and $ E_2$ lie along the x-axis and the
              y-axis,respectively. Let S be the circle $x^2+(y-1)^2=2$. The
		straight line $x+y=3$ touches the curves S, $E_1$ and $E_2$ at $\vec{P}$, $\vec{Q}$
		and $\vec{R}$ respectively. Suppose that $PQ=PR$=$\frac{2\sqrt{2}}{3}$. If $e_1$ and
              $e_2$ are the eccentricities of $E_1$ and $E_2$, respectively, then the 
              correct expression(s) is (are) 
	        
		\hfill(JEE ADV.2015)
		
		 \begin{enumerate}
			\item $e_1^2+e_2^2=\frac{43}{40}$
			\item $e_1e_2=\frac{\sqrt{7}}{2\sqrt{10}}$
			\item $\abs{ e_1^2-e_2^2}=\frac{5}{8}$
			\item $e_1e_2=\frac{\sqrt{3}}{4}$ 
		 \end{enumerate}
	\item Consider the hyperbola H:$x^2-y^2=1$ and a circle S with 
		centre $\vec{N}(x_2,0)$. Suppose that H and S touch each other at a 
	      point $\vec{P}(x_1,y_1)$ with $x_1>0$ and $y_1>0$. The common tangent to H and S at $\vec{P}$ intersects the x-axis at point $\vec{M}$. If $(l,m)$ is the centroid of the triangle $PMN$, then correct expressions(s) is(are)
	      
	      \hfill(JEE ADV.2015)
	      
	       \begin{enumerate}
		      \item $\frac{dl}{dx_1}=1-\frac{1}{3x^2}$ for $x_1>1$ 
		      \item $\frac{dm}{dx_1}=\frac{x_1}{3\sqrt{x_1^2-1}}$ for $x_1>1$ 
		      \item $\frac{dl}{dx_1}=1+\frac{1}{3x^2}$ for $x_1>1$
		      \item $\frac{dm}{dy_1}=\frac{1}{3}$ for $y_1>0$ 
	       \end{enumerate}
      \item The circle $C_1$:$x^2+y^2=3$, with centre at $\vec{O}$, intersects the parabola $x^2=2y$ at the point $\vec{P}$ in the first quadrant. Let the tangent to the circle $C_1$, at $\vec{P}$ touches other two circles $C_2$ and $C_3$ at $\vec{R_2}$ and $\vec{R_3}$, respectively. Suppose $C_2$ and $C_3$ have equal radii $2\sqrt{3}$ and the centres $\vec{Q_2}$ and $\vec{Q_3}$,respectively. If $\vec{Q_2}$ and $\vec{Q_3}$ lie on the y-axis, then 

	      \hfill(JEE ADV.2016)
	      
	       \begin{enumerate}
		      \item $Q_2Q_3=12$
		      \item $R_2R_3=4\sqrt{6}$
		      \item area of the triangle $OR_2R_3$ is $6\sqrt{2}$
		      \item area of the triangle $PQ_2Q_3$ is $4\sqrt{2}$
	       \end{enumerate}
      \item Let $\vec{P}$ be the point on the parabola $y^2=4x$ which is at the shortest distance from the center S of the circle $x^2+y^2-4x-16y+64=0$. Let $\vec{Q}$ be the point on the circle
	      dividing the line segment $SP$ internally. Then 
	      \hfill(JEE ADV.2016)
	      
	       \begin{enumerate}
		      \item $SP=2\sqrt{5}$
		      \item $SQ:QP=(\sqrt{5}+1):2$
		      \item the x-intercept of the normal to the parabola at $\vec{P}$ is $6$
		      \item the slope of the tangent to the circle at $\vec{Q}$ is $\frac{1}{2}$
		      
	       \end{enumerate}
      \item If $2x-y+1=0$ is a tangent to the hyperbola $\frac{x^2}{a^2}-\frac{y^2}{16}=1$ then which of the following cannot be sides of a right angled triangle? 
	      \hfill(JEE ADV.2017)
	      
	       \begin{enumerate}
		      \item $a,4,1$
		      \item $a,4,2$
		      \item $2a,8,1$
		      \item $2a,4,1$
	       \end{enumerate}
      \item If a chord, which is not tangent, of the parabola $y^2=16x$ has equation $2x+y=p$, and midpoint $(h,k)$, then which of the following is(are) possible value(s) of $p$, $h$ and $k$? 
	      \hfill(JEE ADV.2017)
	      
	       \begin{enumerate}
		      \item $p=-2,h=2,k=-4$
		      \item $p=-1,h=1,k=3$
		      \item $p=2,h=3,k=-4$
		      \item $p=5,h=4,k=-3$


	       \end{enumerate}
      \item Consider two straight lines, each of which is tangent to both the circle $x^2+y^2=\frac{1}{2}$
	      and the parabola $y^2=4x$. Let these lines intersect at the point $\vec{Q}$. Consider the elipse whose center is at orgin $\vec{O}(0,0)$ and whose semi-major axis is $OQ$.
	      If the length of the minor axis of the elipse is $\sqrt{2}$, then which of the following statement(s) is(are) TRUE? 
	      \hfill(JEE ADV.2018)
	      
	       \begin{enumerate}
		      \item For the elipse, the eccentricity is $\frac{1}{\sqrt{2}}$ and the length of the latus rectum is $1$

		      \item For the elipse, the eccentricity is $\frac{1}{2}$ and the length of the latus rectum is$\frac{1}{2}$
		      \item The area of the region bounded by the elipse between the lines $x=\frac{1}{\sqrt{2}}$ and $x=1$ is $\frac{1}{4\sqrt{2}}(\pi-2)$
		      \item The area of the region bounded by the elipse between the line $x=\frac{1}{\sqrt{2}}$ and $x=1$ is $\frac{1}{16}(\pi-2)$
	       \end{enumerate}



     % \end{enumerate}

