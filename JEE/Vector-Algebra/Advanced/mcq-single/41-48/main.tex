\iffalse
  \title{20. Vector Algebra}
  \author{Niketh Prakash Achanta - EE24BTECH11047}
  \section{mcq-single}
\fi
	\item %41
		Two adjacent sides of a parallelogram ABCD are given by $AB = 2\hat{i}+10\hat{j}+11\hat{k}$ and $AD = \hat{i}+2\hat{j}+2\hat{k}$ \\
		The side AD is rotated by an acute angle $\alpha$ in the plane of the parallelogram so that AD becomes AD$^{\prime}$. If AD$^{\prime}$ makes a right angle with the side AB, then the cosine of the angle $\alpha$ is given by \hfill{\brak{2010}}\\
\begin{enumerate}
	\item $\frac{8}{9}$
	\item $\frac{\sqrt{17}}{9}$
	\item $\frac{1}{9}$
	\item $\frac{4\sqrt{5}}{9}$\\
\end{enumerate}

        \item %42
		Let $\vec{a}=\hat{i}+\hat{j}+\hat{k}$, $\vec{b}=\hat{i}-\hat{j}+\hat{k}$ and $\vec{c}=\hat{i}-\hat{j}-\hat{k}$ be three vectors. A vector $\vec{v}$ in the plane of $\vec{a}$ and $\vec{b}$ , whose projection on $\vec{c}$ is $\frac{1}{\sqrt{3}}$ , is given by \hfill{\brak{2011}}\\
\begin{enumerate}
	\item $\hat{i}-3\hat{j}+3\hat{k}$
	\item $-3\hat{i}-3\hat{j}-\hat{k}$
	\item $3\hat{i}-\hat{j}+3\hat{k}$
	\item $\hat{i}+3\hat{j}-3\hat{k}$\\
\end{enumerate}
       
	 \item %43
		 The point $\vec{P}$ is the intersection of the straight line joining the points $\vec{Q}\brak{2,3,5}$ and $\vec{R}\brak{1,-1,4}$ with the plane $5x-4y-z=1$. If $\vec{S}$ is the foot of the perpendicular drawn from the point $\vec{T}\brak{2,1,4}$ to QR, then the length of the line segment PS is \hfill{\brak{2010}}\\
\begin{enumerate}
	\item $\frac{1}{\sqrt{2}}$           
	\item $\sqrt{2}$                   
        \item $2$           
	\item $2\sqrt{2}$\\ 
\end{enumerate}
\newpage
         \item %44
		 The equation of a plane passing through the line of intersection of the planes $x+2y+3z=2$ and $x-y+z=3$ and at a distance $\frac{2}{\sqrt{3}}$ from the point $\brak{3,1,-1}$ is \hfill{\brak{2012}}\\
\begin{enumerate}
        \item $5x-11y+z=17$           
	\item $\sqrt{2}x+y=3\sqrt{2}-1$                   
	\item $x+y+z=\sqrt{3}$           
	\item $x-\sqrt{2}y=1-\sqrt{2}$\\ 
\end{enumerate}

         \item %45 
		 If $\vec{a}$ and $\vec{b}$ are vectors such that $\abs{\vec{a}+\vec{b}}$=$\sqrt{29}$ and $\vec{a}\times\brak{2\hat{i}+3\hat{j}+4\hat{k}}$ = $\brak{2\hat{i}+3\hat{j}+4\hat{k}}\times\vec{b}$, then a possible value of $\brak{\vec{a}+\vec{b}}\cdot\brak{-7\hat{i}+2\hat{j}+3\hat{k}}$ is \hfill{\brak{2012}}\\
\begin{enumerate}
        \item $0$                             
        \item $3$                           
        \item $4$            
        \item $8$\\          
\end{enumerate}

         \item %46 
		 Let $\vec{P}$ be the image of the point $\brak{3,1,7}$ with respect to the plane $x-y+z=3$. Then the equation of the plane passing through $\vec{P}$ and containing the straight line $\frac{x}{1}=\frac{y}{z}=\frac{z}{1}$ \hfill{\brak{JEE Adv. 2016}}\\
\begin{enumerate}
        \item $x+y-3z=0$                             
        \item $3x+z=0$                           
        \item $x-4y+7z=0$            
        \item $2x-y=0$\\          
\end{enumerate}

         \item %47 
		 The equation of the plane passing through the point $\brak{1,1,1}$ and perpendicular to the planes $2x+y-2z=5$ and $3x-6y-2z=7$, is \hfill{\brak{JEE Adv. 2017}}\\
\begin{enumerate}
        \item $14x+2y-15z=1$                             
        \item $14x-2y+15z=27$                           
        \item $14x+2y+15z=31$            
        \item $-14x+2y+15z=3$\\          
\end{enumerate}

         \item %48 
		 Let $\vec{O}$ be the origin and let PQR be an arbitrary triangle. The point $\vec{S}$ is such that $OP\cdot OQ$+$OR\cdot OS$=$OR\cdot OP$+$OQ\cdot OS$=$OQ\cdot OR$+$OP\cdot OS$\\
Then the triangle PQR has $\vec{S}$ as its \hfill{\brak{JEE Adv. 2017}}\\
\begin{enumerate}
        \item Centroid                             
        \item Circumcentre                           
        \item Incentre            
        \item Orthocenter\\          
\end{enumerate}

 
