\iffalse
\title{Assignment-3}
\author{EE24BTECH11048-NITHIN.K}
\section{mcq-multiple}
\fi
%\begin{enumerate}
%8
	\item Let a and b be two non-collinear unit vectors. If $u=a-\brak{a\cdot b}b$ and $v=a\times b$, then $\abs{v}$ is \hfill{(1999-3 Marks)}
		\begin{enumerate}
			\item $\abs{u}$
			\item $\abs{u} + \abs{u \cdot a}$
			\item $\abs{u} + \abs{u \cdot b}$
			\item $\abs{u} +  u\cdot\brak{a+b}$
		\end{enumerate}
%9
	\item Let $\vec{A}$ be vector parallel to line of intersection of planes $P_1$ and $P_2$. Plane $P_1$ is parallel to the vectors $2\hat{j}+3\hat{k}$ and $4\hat{j}-3\hat{k}$ and that $P_2$ is
		parallel to $\hat{j}-\hat{k}$ and $3\hat{i}+3\hat{j}$, then the angle between vector $\vec{A}$ and a given vector $2\hat{i}+\hat{j}-2\hat{k}$ is \hfill{(2006-5M,-1)}
		\begin{enumerate}
			\item $\frac{\pi}{2}$
			\item $\frac{\pi}{4}$
			\item $\frac{\pi}{6}$
			\item $\frac{3\pi}{4}$
		\end{enumerate}
%10
	\item The vector(s) which is/are coplanar with vectors $\hat{i}+\hat{j}+2\hat{k}$ and $\hat{i}+2\hat{j}+\hat{k}$, and perpendicular to the vector $\hat{i}+\hat{j}+\hat{k}$ is/are \hfill{(2011)}
		\begin{enumerate}
			\item $\hat{j}-\hat{k}$
			\item $\hat{i}+\hat{j}$
			\item $\hat{i}-\hat{j}$
			\item $\hat{j}+\hat{k}$
		\end{enumerate}
%11
	\item If the straight lines $\frac{x-1}{2}=\frac{y+1}{k}=\frac{z}{2}$ and $\frac{x+1}{5}=\frac{y+1}{2}=\frac{z}{k}$ are coplanar,then the plane(s) containing these two lines is(are) \hfill{(2012)}
		\begin{enumerate}
			\item $y+2z=-1$
			\item $y+z=-1$
			\item $y-z=-1$
			\item $y-2z=-1$
		\end{enumerate}
%12
	\item A line L passing through the origin is perpendicular to the lines
		\begin{align*}
			l_1 : \brak{3+t}\hat{i}+\brak{1+2t}\hat{j}+\brak{4+2t}\hat{k}, -\infty<t<\infty \\
			l_2 : \brak{3+2s}\hat{i}+\brak{3+2s}\hat{j}+\brak{2+s}\hat{k},-\infty<s<\infty 
		\end{align*}
		Then, the coordinate(s) of the point(s) on $l_2$ at a distance of $\sqrt{17}$ from the point of intersection of L and $l_1$ is(are) \hfill{(JEE Adv.2013)}
		\begin{enumerate}
			\item $\brak{\frac{7}{3},\frac{7}{3},\frac{5}{3}}$
			\item $\brak{-1,-1,0}$
			\item $\brak{1,1,1}$
			\item $\brak{\frac{7}{9},\frac{7}{9},\frac{8}{9}}$
		\end{enumerate}
%13
	\item Two lines $L_1: x=5,\frac{y}{3-\alpha}=\frac{z}{-2}$ and $L_2: x=\alpha,\frac{y}{-1}=\frac{z}{2-\alpha}$ are coplanar. Then $\alpha$ can take value(s) \hfill{(JEE Adv.2013)}
		\begin{enumerate}
			\item 1
			\item 2 
			\item 3
			\item 4
		\end{enumerate}
%14
	\item Let $\vec{x},\vec{y}$ and $\vec{z}$ be three vectors each of magnitude $\sqrt{2}$ and the angle between each pair of them is $\frac{\pi}{3}$. If $\vec{a}$ is a non-zero vector perpendicular
		to $\vec{x}$ and $\vec{y}\times \vec{z}$ and $\vec{b}$ is a non-zero vector perpendicular to $\vec{y}$ and $\vec{z}\times \vec{x}$, then \hfill{(JEE Adv.2014)}
		\begin{enumerate}
			\item $\vec{b}=\brak{\vec{b} \cdot \vec{z}}\brak{\vec{z}-\vec{x}}$
			\item $\vec{a}=\brak{\vec{a} \cdot \vec{y}}\brak{\vec{y}-\vec{z}}$
			\item $\vec{a}\cdot\vec{b}=-\brak{\vec{a} \cdot \vec{y}}\brak{\vec{b}\cdot\vec{z}}$
			\item $\vec{a}=-\brak{\vec{a} \cdot \vec{y}}\brak{\vec{z}-\vec{y}}$
		\end{enumerate}
%15
	\item From a point $P\brak{\lambda,\lambda,\lambda}$, perpendicular PQ and PR are drawn respectively on the lines $y=x,z=1$ and $y=-x,z=-1$. If P is such that $\angle QPR$ is a right angle, then
		the possible value(s) of $\lambda$ is/(are) \hfill{(JEE Adv.2014)}
		\begin{enumerate}
			\item $\sqrt{2}$
			\item 1 
			\item -1 
			\item -$\sqrt{2}$
		\end{enumerate}
%16
	\item In $R^3$, consider the planes $P_1:y=0$ and $P_2:x+z-1$.Let $P_3$ be the plane different from $P_1$ and $P_2$ which passes through the intersection of $P_1$ and $P_2$. If the distance of the
		point\brak{0,1,0} from $P_3$ is 1 and the distance of point $\brak{\alpha,\beta,\gamma}$ from $P_3$ is 2,  then which of the following relation is(are) true \hfill{(JEE Adv.2015)}
		\begin{enumerate}
			\item $2\alpha+\beta+2\gamma+2=0$
			\item $2\alpha-\beta+2\gamma+4=0$
			\item $2\alpha+\beta+2\gamma-10=0$
			\item $2\alpha-\beta+2\gamma-8=0$
		\end{enumerate}
%17
	\item In $R^3$, let L be a straight line passing through the origin. Suppose that all the points on L are at a costant distance from two planes $P_1:x+2y-z+1=0$ and $P_2:2x-y+z-1=0$. Let M be the
		locus of the feet of the perpendicular drawn from the points on L to the plane $P_1$. Which of the following points lie(s) on M? \\ \hfill{(JEE Adv.2015)}
		\begin{enumerate}
			\item $\brak{0,-\frac{5}{6},-\frac{2}{3}}$
			\item $\brak{-\frac{1}{6},-\frac{1}{3},\frac{1}{6}}$
			\item $\brak{-\frac{5}{6},0,\frac{2}{3}}$
			\item $\brak{-\frac{1}{3},0,\frac{2}{3}}$
		\end{enumerate}
%18
	\item Let $\triangle PQR$ be a triangle. Let $\vec{a}=\vec{QR}$, $\vec{b}=\vec{RP}$ and $\vec{c}=\vec{PQ}$. If $\abs{\vec{a}}=12$, $\abs{\vec{b}}=4\sqrt{3}$, $\vec{b}
		\cdot \vec{c}=24$, then which of the following is(are)true? \hfill{(JEE Adv.2015)}
		\begin{enumerate}
			\item $\frac{\abs{\vec{c}}^2}{2}-\abs{\vec{a}}=12$
			\item $\frac{\abs{\vec{c}}^2}{2}+\abs{\vec{a}}=30$
			\item $\abs{\vec{a}\times \vec{b}+\vec{c}\times \vec{a}}=48\sqrt{3}$
			\item $\vec{a}\cdot\vec{b}=-72$
		\end{enumerate}
%19
	\item Consider a pyramid OPQRS located in the first octant $(x\geq 0,y\geq 0,z\geq 0)$ with O as origin, and OP and OR along the x-axis and the y-axis respectively. The base OPQR of the pyramid is
		a square with OP=3. The point S is directly above the mid-point, T of diagonal OQ such that TS=3. Then \hfill{(JEE Adv.2016)}
		\begin{enumerate}
			\item the acute angle between OQ and OS is $\frac{\pi}{3}$
			\item the equation of the plane containg the triangle OQS is $x-y=0$
			\item the length of the perpendicular from P to the plane containing the triangle OQS is $\frac{3}{\sqrt{2}}$
			\item the perpendicular distance from O to the staright line containing RS is $\sqrt{\frac{15}{2}}$
		\end{enumerate}
%20
	\item Let $\hat{u}=u_1\hat{i}+u_2\hat{j}+u_3\hat{k}$ be a unit vector in $R^3$ and $\hat{w}=\frac{1}{\sqrt{6}}\brak{\hat{i}+\hat{j}+2\hat{k}}$. Given that there exists a vector $\vec{v}$ in $R^3$ 
		such that $\abs{\hat{u} \times \vec{v}}=1$ and $\hat{w}\brak{\hat{u}\times \vec{v}}=1$. Which of the following statement(s) is(are) correct? \hfill{(JEE Adv.2016)}
		\begin{enumerate}
			\item there is exactly one choice for such $\vec{v}$
			\item There are infinitely many choices for such $\vec{v}$
			\item If $\hat{u}$ lies in the xy-plane then $\abs{u_1}= \abs{u_2}$
			\item If $\hat{u}$ lies in the xz-plane then $2\abs{u_1}=\abs{u_3}$
		\end{enumerate}
%21
	\item Let $P_1:2x+y-z=3$ and $P_2:x+2y+z=2$ be two planes. Then,which of the following statement(s) is(are) TRUE? \hfill{(JEE Adv.2018)}
		\begin{enumerate}
			\item The line of intersection of$P_1$ and $P_2$ has direction ratios 1,2,-1
			\item The line $\frac{3x-4}{9}=\frac{1-3y}{9}=\frac{z}{3}$ \\ is perpendicular to the line of intersection of $P_1$ and $P_2$
			\item The acute angle between $P_1$ and $P_2$ is $60\degree$.
			\item If $P_3$ is the plane passing through the point \brak{4,2,-2} and perpendicular to the line of intersection of $P_1$ and $P_2$,then the distance of the point \brak{2,1,1}
				from the plane $P_3$ is $\frac{2}{\sqrt{3}}$
		\end{enumerate}
%22
	\item Let $L_1$ and $L_2$ denote the lines\\ $\vec{r}=\hat{i}+\lambda\brak{-\hat{i}+2\hat{j}+2\hat{k}}, \lambda \in R$ and \\
		$\vec{r}=\mu \brak{2\hat{i}-\hat{j}+2\hat{k}}, \mu \in R$\\ respectively. If $L_3$ is a line which is perpendicular to both $L_1$ and $L_2$ and cuts both of them, then which of 
		the following option describe(s) $L_3$? \hfill{(JEE Adv.2019)}
		\begin{enumerate}
			\item $\vec{r}=\frac{2}{9}\brak{4\hat{i}+\hat{j}+\hat{k}}+t\brak{2\hat{i}+2\hat{j}-\hat{k}},t \in R$
			\item $\vec{r}=\frac{2}{9}\brak{2\hat{i}-\hat{j}+2\hat{k}}+t\brak{2\hat{i}+2\hat{j}-\hat{k}}, t \in R$
			\item $\vec{r}=t\brak{2\hat{i}+2\hat{j}-\hat{k}},t \in R$
			\item $\vec{r}=\frac{1}{3}\brak{2\hat{i}+\hat{k}}+t\brak{2\hat{i}+2\hat{j}-\hat{k}},t \in R$
		\end{enumerate}
%\end{enumerate}
